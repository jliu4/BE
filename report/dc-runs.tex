\documentclass{article}
\usepackage{graphicx}
\title{DC RUNS}
\author{Jin Liu}
\setlength\parindent{0pt}
\begin{document}
\maketitle

We have done five runs of applying DC power to estimate the pulses at the surface of the core. Table 1 lists all runs time, helium or hydrogen, and the core. 
Assuming there is a linear relationship between Heater Power drop vs. DC Power at each core temperature. Figure 1 presents the slopes (M) vs. the core temperatures for all runs. 
Figure 2 through Figure 6 are the Heater Power, Temperature, DC Power and DC Volt vs. time. 
    
\begin{table}
[h]
\centering
\caption{DC Runs}
\begin{tabular}{|c|c|c|}
\hline
08/29/2016 & H2 & ipb2-27b\\ \hline
10/03/2016 & He & ipb1-29b\\  \hline
10/12/2016 & He & ipb1-29b\\ \hline
10/15/2016 & H2 & sri-ipb2-27b\\  \hline
10/25/2016 & H2 & ipb1-29b\\  \hline
\end{tabular}
\end{table} 
  
\begin{table}
[h]
\centering
\caption{DC Runs}
\begin{tabular}{|c|c|c|c|c|c\}
\hline
08/29/2016 & H2 & ipb2-27b\\ \hline
10/03/2016 & He & ipb1-29b\\  \hline
10/12/2016 & He & ipb1-29b\\ \hline
10/15/2016 & H2 & sri-ipb2-27b\\  \hline
10/25/2016 & H2 & ipb1-29b\\  \hline
\end{tabular}
\end{table}   
\begin{figure}
[h]
\begin{center}
\includegraphics[scale=0.7]{hourvsTempPower.p} 
\caption{Inner Core Temperature and Heat Power vs. Running Hours}%
\end{center}
\end{figure}



\end{document}
