\documentclass{article}
\usepackage{graphicx}
\usepackage{amsmath}%
\setcounter{MaxMatrixCols}{30}%
\usepackage{amsfonts}%
\usepackage{amssymb}%
\title{IPB Reactor Observation Note}
\author{Jin Liu}
\date{January 25, 2017}
\date{February 17, 2017}
\setlength\parindent{0pt}
\begin{document}
\maketitle

Definition\\

$Hpdrop:$ \ heater power drop after power deposit to the core in watts\\

$V_{1}:$ voltage RMS measured at the core entrance when $Q$-pulse\\

$V_{2}:$ voltage RMS measured at the core exit when $Q$-pulse\\

$V_{3}:$ voltage RMS measured across the RF termination resistor at the end of the transmission line. The termination resistors are mounted in a copper block that is water cooled . It has constant RF impedance in the freq range we are operating in. With this method we can measure the pulse current directly by measuring $V_{3}$ and knowing the $R_{term}$ resistance, $I = V_{3} / R_{term}$ \\

$P:$ power deposit to the core either by $DC$ or $Q$-pulse in watts\\
in $Q$-pulse 

\begin{equation}
P=\frac{(V_{1}-V_{2})*V_{3}}{R_{term}} \label{1}%
\end{equation}
%

$V^{2}=(V_{1}-V_{2})^{2}$ when $Q$-pulse or voltage drop when $DC$ \\


Observation\\

\begin{equation}
R=\frac{V^{2}}{P}\ [volts^{2}/watts], [volts^{2}/watts]=[ohms]\label{3}%
\end{equation}
%
Where R is constant at any given core temperature for power DC or $Q$-pulse, gas helium or hydrogen.\\
\begin{equation}
M=\frac{Hpdrop}{P}\ \label{4}%
\end{equation}
%
Where M is constant at any given core temperature for power DC or $Q$-pulse, gas helium or hydrogen.\\


\end{document}
